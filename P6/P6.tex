\documentclass[a4paper,12pt]{article}
\usepackage{CJKutf8}
\usepackage{amsthm}
\usepackage{amsmath}
\usepackage{amssymb}
\usepackage{geometry}
\usepackage{tikz}
\usetikzlibrary{chains}
\usepackage{subfigure}

% 边距
\geometry{left=2.0cm,right=2.0cm,top=2.0cm,bottom=3.0cm}

\newtheorem{theorem}{Theorem}
\newtheorem{lemma}[theorem]{Lemma}
\newtheorem{proposition}[theorem]{Proposition}
\newtheorem{corollary}[theorem]{Corollary}
\newtheorem{exercise}{Exercise}
\newtheorem*{solution}{Solution}
\newtheorem{definition}{Definition}
\theoremstyle{definition}

\makeatletter \renewenvironment{proof}[1][Proof] {\par\pushQED{\qed}\normalfont\topsep6\p@\@plus6\p@\relax\trivlist\item[\hskip\labelsep\bfseries#1\@addpunct{.}]\ignorespaces}{\popQED\endtrivlist\@endpefalse} \makeatother
\makeatletter
\renewenvironment{solution}[1][Solution] {\par\pushQED{\qed}\normalfont\topsep6\p@\@plus6\p@\relax\trivlist\item[\hskip\labelsep\bfseries#1\@addpunct{.}]\ignorespaces}{\popQED\endtrivlist\@endpefalse} \makeatother

% 大题
\newenvironment{problems}{\begin{list}{}{\renewcommand{\makelabel}[1]{\textbf{##1}\hfil}}}{\end{list}}

% 小题
\newenvironment{steps}{\begin{list}{}{\renewcommand{\makelabel}[1]{\textbf{##1}\hfil}}}{\end{list}}

% 标题
\title{\small \underline{Mathematical Foundations of Computer Science}\\\Large Project 6}
\author{Log Creative\\\small Student ID: }
\date{\today}

\begin{document}
\maketitle

\noindent\textbf{Warmups}

\begin{problems}

    \item[7] Let $\nabla f(x)=f(x)-f(x-1)$. What is $\nabla (x^{\overline{m}})$?
    \begin{solution} 
    \begin{align*}
        \nabla (x^{\overline{m}}) &= \nabla (x(x+1)\cdots(x+m-1))\\
        &=x(x+1)\cdots(x+m-1) - (x-1)x\cdots(x+m-2)\\
        &=x(x+1)\cdots(x+m-2)\left[(x+m-1)-(x-1)\right]\\
        &=mx^{\overline{m-1}}
    \end{align*} 
    \end{solution}
    \item[8] What is the value of $0^{\underline{m}}$, when $m$ is a given integer?
    \begin{solution}
        \begin{description}
            \item[Case 1: $m>0$.]  \begin{equation*}
                0^{\underline{m}} = 0\cdot (-1)^{\underline{m-1}} = 0
            \end{equation*}
            \item[Case 2: $m=0$.] \begin{equation*}
                0^{\underline{0}} = 1
            \end{equation*}
            which is defined as the empty product.
            \item[Case 3: $m<0$.] \begin{align*}
                1 = 0^{\underline{0}} &= 0^{\underline{m}}(-m)^{\underline{-m}} = 0^{\underline{m}}|m|!\\
                0^{\underline{m}}&=\frac{1}{|m|!}
            \end{align*}
        \end{description}
    \end{solution} 
    \item[9] What  is  the  law  of  exponents  for  rising  factorial  powers,  analogous  to (2.52)? Use this to define $x^{\overline{-n}}$.
    \begin{solution}
        \begin{equation*}
            x^{\overline{m+n}} = x(x+1)\cdots(x+m-1)\cdots(x+m)\cdots(x+m+n-1) = x^{\overline{m}}(x+m)^{\overline{n}}
        \end{equation*}
        \begin{equation*}
            1 = x^{\overline{0}}= x^{\overline{(-n)+n}} = x^{\overline{-n}}(x-n)^{\overline{n}}
        \end{equation*}
        \begin{equation*}
            x^{\overline{-n}} = \frac{1}{(x-n)^{\overline{n}}} = \frac{1}{(x-n)(x-n+1)\cdots(x-1)}
        \end{equation*}
    \end{solution} 
    \item[10] The text derives the following formula for the difference of a product:
    \begin{equation*}
        \Delta(uv) = u\Delta v + Ev\Delta u
    \end{equation*}
    How can this formula be correct, when the left-hand side is symmetric with respect to $u$ and $v$ but the right-hand side is not?
    \begin{solution}
        Because the following formula could also be correct with tht respect of $u$:
        \begin{align*}
            \Delta(uv) &= u(x+1)v(x+1) - u(x)v(x)\\
            &=u(x+1)v(x+1) - u(x+1)v(x) + u(x+1)v(x) - u(x)v(x)\\
            &=u(x+1)\Delta v + v\Delta u\\
            &=Eu\Delta v + v\Delta u
        \end{align*}
        It is an inner layer of symmetric.
        \begin{equation*}
            Eu\Delta v + v\Delta u = u\Delta v + Ev\Delta u
        \end{equation*}
    \end{solution}
\end{problems}

\noindent\textbf{Basics}

\begin{problems}
    \item[14] Evaluate $\sum_{k=1}^n 2^k$ by rewriting it as the multiple sum $\sum_{1\leq j \leq k\leq n}2^k$.
    \begin{align*}
        \sum_{1\leq j \leq k\leq n}2^k &= \sum_{1\leq j\leq n}\sum_{j\leq k\leq n}2^k \\&= \sum_{1\leq j\leq n}\left[(2^1+2^2+\cdots+2^n) - (2^1+2^2+\cdots +2^{j-1})\right] \\&= \sum_{1\leq j\leq n}\left[(2^{n+1}-2)-(2^j-2)\right] \\&= n2^{n+1}-(2^{n+1}-2)
    \end{align*} 
\end{problems}

\end{document}
