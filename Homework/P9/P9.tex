\documentclass[a4paper,12pt]{article}
\usepackage{CJKutf8}
\usepackage{amsthm}
\usepackage{amsmath}
\usepackage{amssymb}
\usepackage{geometry}
\usepackage{tikz}
\usetikzlibrary{chains}
\usepackage{subfigure}
\usepackage{enumerate}

% 边距
\geometry{left=2.0cm,right=2.0cm,top=2.0cm,bottom=3.0cm}

\newtheorem{theorem}{Theorem}
\newtheorem{lemma}[theorem]{Lemma}
\newtheorem{proposition}[theorem]{Proposition}
\newtheorem{corollary}[theorem]{Corollary}
\newtheorem{exercise}{Exercise}
\newtheorem*{solution}{Solution}
\newtheorem{definition}{Definition}
\theoremstyle{definition}

\makeatletter \renewenvironment{proof}[1][Proof] {\par\pushQED{\qed}\normalfont\topsep6\p@\@plus6\p@\relax\trivlist\item[\hskip\labelsep\bfseries#1\@addpunct{.}]\ignorespaces}{\popQED\endtrivlist\@endpefalse} \makeatother
\makeatletter
\renewenvironment{solution}[1][Solution] {\par\pushQED{\qed}\normalfont\topsep6\p@\@plus6\p@\relax\trivlist\item[\hskip\labelsep\bfseries#1\@addpunct{.}]\ignorespaces}{\popQED\endtrivlist\@endpefalse} \makeatother

% 大题
\newenvironment{problems}{\begin{list}{}{\renewcommand{\makelabel}[1]{\textbf{##1}\hfil}}}{\end{list}}

% 小题
\newenvironment{steps}{\begin{list}{}{\renewcommand{\makelabel}[1]{\textbf{##1}\hfil}}}{\end{list}}

% 标题
\title{\small \underline{Mathematical Foundations of Computer Science}\\\Large Project 9}
\author{Zilong Li\\\small Student ID: 518070910095}
\date{\today}

\begin{document}
\maketitle

\noindent\textbf{Warmups}

\begin{problems}
    \item[1] What  is  the  smallest  positive  integer  that  has  exactly k divisors,  for $1\leq k\leq 6$?
    \item[2] Prove that $\gcd (m, n)\cdot\text{lcm} (m, n)=m\cdot n$, and use this identity to express lcm($m, n$) in terms of lcm($n\bmod m, m$), when $n \bmod m\neq 0$. Hint: Use (4.12), (4.14), and (4.15).
    \item[3] Let $\pi(x)$ be the number of primes not exceeding $x$.  Prove or disprove: $\pi(x) -\pi(x-1) =$ [$x$ is prime].
    \item[4] What would happen if the Stern-Brocot construction started with the five fractions $\left(\frac{0}{1},\frac{1}{0},\frac{0}{-1},\frac{-1}{0},\frac{0}{1}\right)$ instead of with $\left(\frac{0}{1},\frac{1}{0}\right)$?
    \item[5] Find simple formulas for $L^k$ and $R^k$, when $L$ and $R$ are the 2$\times$2 matrices of (4.33).
    \item[6] What does `$a \equiv b\pmod 0$' mean?
    \item[7] Ten people numbered 1 to 10 are lined up in a circle as in the Josephus problem,  and every mth person is executed.  (The value of $m$ may bemuch larger than 10.)  Prove that the first three people to go cannot be 10,$k$, and $k+1$(in this order), for any $k$.
\end{problems}

\noindent\textbf{Basics}

\begin{problems}
    \item[14] Prove or disprove:
    \begin{enumerate}[a]
        \item gcd($km, kn$) =$k$gcd($m, n$)
        \item lcm($km, kn$) =$k$lcm($m, n$)
    \end{enumerate} 
    \item[15] Does every prime occur as a factor of some Euclid number $e_n$?
\end{problems}

\end{document}