\documentclass[a4paper,12pt]{article}
\usepackage{CJKutf8}
\usepackage{amsthm}
\usepackage{amsmath}
\usepackage{amssymb}
\usepackage{geometry}
\usepackage{tikz}
\usetikzlibrary{chains}
\usepackage{subfigure}

% 边距
\geometry{left=2.0cm,right=2.0cm,top=2.0cm,bottom=3.0cm}

\newtheorem{theorem}{Theorem}
\newtheorem{lemma}[theorem]{Lemma}
\newtheorem{proposition}[theorem]{Proposition}
\newtheorem{corollary}[theorem]{Corollary}
\newtheorem{exercise}{Exercise}
\newtheorem*{solution}{Solution}
\newtheorem{definition}{Definition}
\theoremstyle{definition}

\makeatletter \renewenvironment{proof}[1][Proof] {\par\pushQED{\qed}\normalfont\topsep6\p@\@plus6\p@\relax\trivlist\item[\hskip\labelsep\bfseries#1\@addpunct{.}]\ignorespaces}{\popQED\endtrivlist\@endpefalse} \makeatother
\makeatletter
\renewenvironment{solution}[1][Solution] {\par\pushQED{\qed}\normalfont\topsep6\p@\@plus6\p@\relax\trivlist\item[\hskip\labelsep\bfseries#1\@addpunct{.}]\ignorespaces}{\popQED\endtrivlist\@endpefalse} \makeatother

% 大题
\newenvironment{problems}{\begin{list}{}{\renewcommand{\makelabel}[1]{\textbf{##1}\hfil}}}{\end{list}}

% 小题
\newenvironment{steps}{\begin{list}{}{\renewcommand{\makelabel}[1]{\textbf{##1}\hfil}}}{\end{list}}

% 标题
\title{\small \underline{Mathematical Foundations of Computer Science}\\\Large Project 8}
\author{Zilong Li\\\small Student ID: 518070910095}
\date{\today}

\begin{document}
\maketitle

\noindent\textbf{Warmups}

\begin{problems}
    \item[6] Can something interesting be said about $\lfloor f(x)\rfloor$ when $f(x)$ is a continuous, monotonically \emph{decreasing} function that takes integer values only when $x$ is an integer?
    \item[7] Solve the recurrence
    \begin{align*}
        X_n&=n,&&\text{for }0\leq n<m\;\\
        X_n&=X_{n-m}+1,&&\text{for }n\geq m.
    \end{align*} 
    \item[8] Prove the \emph{Dirichlet box} principle:  If $n$ objects are put into $m$ boxes, some box must contain $\geq \lceil n/m\rceil$ objects, and some box must contain $\leq \lfloor n/m\rfloor$.
    \item[9] Egyptian mathematicians in 1800 \textsc{b.c.} represented rational numbers between 0 and 1 as sums of unit fractions $1/x_1+\cdots+1/x_k$, where the $x$'s were distinct positive integers.  For example, they wrote $1/3+1/15$ instead of $2/5$.  Prove that it is always possible to do this in a systematic way:  If $0 < m/n < 1$, then
    \begin{equation*}
        \frac{m}{n} =\frac{1}{q} + \left\{\text{representation of }\frac{m}{n}-\frac{1}{q}\right\},\quad q=\left\lceil\frac{n}{m}\right\rceil
    \end{equation*} 
    (This is \emph{Fibonacci's algorithm}, due to Leonardo Fibonacci, \textsc{a.d.} 1202.)
\end{problems}

\noindent\textbf{Basics}

\begin{problems}
    \item[10]Show that the expression
    \begin{equation*}
        \left\lceil\frac{2x+1}{2}\right\rceil - \left\lceil\frac{2x+1}{4}\right\rceil + \left\lfloor\frac{2x+1}{4}\right\rfloor
    \end{equation*} 
    is always either $\lfloor x \rfloor$ or $\lceil x \rceil$. In what circumstances does each case arise?
    \item[11] Give details of the proof alluded to in the text,  that the open interval$(\alpha .. \alpha)$contains exactly $\lceil \beta\rceil-\lfloor \alpha\rfloor-1$ integers when $\alpha < \beta$.  Why doesthe case $\alpha=\beta$ have to be excluded in order to make the proof correct?
    \item[12]Prove that
    \begin{equation*}
        \left\lceil \frac{n}{m}\right\rceil = \left\lfloor\frac{n+m-1}{m}\right\rfloor
    \end{equation*} 
    for  all  integers $n$ and  all  positive  integers $m$.   [This  identity  gives  usanother way to convert ceilings to floors and vice versa, instead of using the reflective law (3.4).]
    \item[13]  Let $\alpha$ and $\beta$ be positive real numbers.  Prove that Spec($\alpha$) and Spec($\beta$) partition the positive integers if and only if $\alpha$ and $\beta$ are irrational and $1/\alpha+1/\beta=1$.
\end{problems}

\end{document}