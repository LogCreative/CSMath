\documentclass[a4paper,12pt]{article}
\usepackage{CJKutf8}
\usepackage{amsthm}
\usepackage{amsmath}
\usepackage{amssymb}
\usepackage{geometry}
\usepackage{tikz}
\usetikzlibrary{chains}
\usepackage{subfigure}

% 边距
\geometry{left=2.0cm,right=2.0cm,top=2.0cm,bottom=3.0cm}

\newtheorem{theorem}{Theorem}
\newtheorem{lemma}[theorem]{Lemma}
\newtheorem{proposition}[theorem]{Proposition}
\newtheorem{corollary}[theorem]{Corollary}
\newtheorem{exercise}{Exercise}
\newtheorem*{solution}{Solution}
\newtheorem{definition}{Definition}
\theoremstyle{definition}

\makeatletter \renewenvironment{proof}[1][Proof] {\par\pushQED{\qed}\normalfont\topsep6\p@\@plus6\p@\relax\trivlist\item[\hskip\labelsep\bfseries#1\@addpunct{.}]\ignorespaces}{\popQED\endtrivlist\@endpefalse} \makeatother
\makeatletter
\renewenvironment{solution}[1][Solution] {\par\pushQED{\qed}\normalfont\topsep6\p@\@plus6\p@\relax\trivlist\item[\hskip\labelsep\bfseries#1\@addpunct{.}]\ignorespaces}{\popQED\endtrivlist\@endpefalse} \makeatother

% 大题
\newenvironment{problems}{\begin{list}{}{\renewcommand{\makelabel}[1]{\textbf{##1}\hfil}}}{\end{list}}

% 小题
\newenvironment{steps}{\begin{list}{}{\renewcommand{\makelabel}[1]{\textbf{##1}\hfil}}}{\end{list}}

% 标题
\title{\small \underline{Mathematical Foundations of Computer Science}\\\Large Project 4}
\author{Zilong Li\\\small Student ID: 518070910095}
\date{\today}

\begin{document}
\maketitle

\noindent\textbf{Warmups}

\begin{problems}
    \item[1] What does the notation
    \begin{equation*}
        \sum_{i=4}^0 q_i
    \end{equation*} 
    mean?
    \begin{solution}
        In a programmer's perspective, this notation could be interpreted as the decreasing order on the index:
        \begin{equation*}
            \sum_{i=4}^0 q_i = q_4 + q_3 + q_2 + q_1 + q_0 = \sum_{i=0}^4 q_k
        \end{equation*}
        
        But in mathematics, the formula also holds:
        \begin{equation*}
            \sum_{i=4}^0 q_i = \sum_{4\leq i \leq 0} q_i = \sum_{\varnothing} q_i = 0
        \end{equation*}

        However, if $\sum_{k\leq n}q_k$ and $\sum_{k<m} q_k$ are finite sums, the following interpretation also holds if the summation is regarded generally:
        \begin{equation*}
            \sum_{i=4}^0 q_i = \sum_{i\leq 0} q_i - \sum_{i<4} q_i = -q_1-q_2-q_3
        \end{equation*}

        The interpretation is different when the definition differs, one should always use an increasing order on index to avoid misinterpretations.
    \end{solution}
    \item[2] Simplify the expression $x\cdot ([x>0]-[x<0])$. 
    \begin{solution}
        According to Iverson's convention, 
        \begin{equation*}
            x\cdot([x>0]-[x<0]) = \left\{\begin{aligned}
                &x\cdot 1,&&x>0,\\
                &0\cdot 0,&&x=0,\\
                &x\cdot(-1),&&x<0\\
            \end{aligned}\right. = |x|
        \end{equation*}
    \end{solution} 
    \item[3] Demonstrate your understanding of $\sum$-notation by writing out the sums
    \begin{equation*}
        \sum_{0\leq k\leq 5}a_k \text{ and } \sum_{0\leq k^2\leq 5}a_{k^2}
    \end{equation*} 
    in full. (Watch out the second sum is a bit tricky.)
    \begin{solution}
        \begin{equation*}
            \sum_{0\leq k\leq 5}a_k = a_0 + a_1 + a_2 + a_3 + a_4 + a_5
        \end{equation*}

        The second notation here impiles:
        \begin{equation*}
            0\leq k^2\leq 5 \Rightarrow k = \{0,1,2,-1,-2\}
        \end{equation*}
        So, the summation follows:
        \begin{equation*}
            \sum_{0\leq k^2\leq 5}a_{k^2} = a_0 + 2a_1 + 2a_2
        \end{equation*}

    \end{solution}
    \item[4] Express the triple sum
    \begin{equation*}
        \sum_{1\leq i<j<k\leq 4}a_{ijk}
    \end{equation*} 
    as a three-fold summation (with three $\sum$'s),
    \begin{steps}
        \item[a] summing first on $k$, then $j$, then $i$;
        \item[b] summing first on $i$, then $j$, then $k$. 
    \end{steps}
    Also write your triple sums out in full without the $\sum$-notation, using parentheses to show what is being added together first.
    \begin{solution}
        (This solution is broken.)
        % \begin{align*}
        %     \sum_{1\leq i<j<k\leq 4}a_{ijk} &= \sum_{i=1}^4\sum_{j=1}^4\sum_{k=1}^4 a_{ijk} \\&= \sum_{i=1}^4\sum_{j=1}^4 (a_{ij,1} + a_{ij,2} + a_{ij,3} + a_{ij,4}) \\
        %     &= \sum_{i=1}^4[(a_{i,1,1} + a_{i,1,2} + a_{i,1,3} + a_{i,1,4})+(a_{i,2,1} + a_{i,2,2} + a_{i,2,3} + a_{i,2,4})\\&+(a_{i,3,1} + a_{i,3,2} + a_{i,3,3} + a_{i,3,4})+(a_{i,4,1} + a_{i,4,2} + a_{i,4,3} + a_{i,4,4})]\\
        %     &= [(a_{1,1,1} + a_{1,1,2} + a_{1,1,3} + a_{1,1,4})+(a_{1,2,1} + a_{1,2,2} + a_{1,2,3} + a_{1,2,4})\\&+(a_{1,3,1} + a_{1,3,2} + a_{1,3,3} + a_{1,3,4})+(a_{1,4,1} + a_{1,4,2} + a_{1,4,3} + a_{1,4,4})]\\&+[(a_{2,1,1} + a_{2,1,2} + a_{2,1,3} + a_{2,1,4})+(a_{2,2,1} + a_{2,2,2} + a_{2,2,3} + a_{2,2,4})\\&+(a_{2,3,1} + a_{2,3,2} + a_{2,3,3} + a_{2,3,4})+(a_{2,4,1} + a_{2,4,2} + a_{2,4,3} + a_{2,4,4})]\\&+[(a_{3,1,1} + a_{3,1,2} + a_{3,1,3} + a_{3,1,4})+(a_{3,2,1} + a_{3,2,2} + a_{3,2,3} + a_{3,2,4})\\&+(a_{3,3,1} + a_{3,3,2} + a_{3,3,3} + a_{3,3,4})+(a_{3,4,1} + a_{3,4,2} + a_{3,4,3} + a_{3,4,4})]\\&+[(a_{4,1,1} + a_{4,1,2} + a_{4,1,3} + a_{4,1,4})+(a_{4,2,1} + a_{4,2,2} + a_{4,2,3} + a_{4,2,4})\\&+(a_{4,3,1} + a_{4,3,2} + a_{4,3,3} + a_{4,3,4})+(a_{4,4,1} + a_{4,4,2} + a_{4,4,3} + a_{4,4,4})]
        % \end{align*}
        % \begin{align*}
        %     \sum_{1\leq i<j<k\leq 4}a_{ijk} &= \sum_{k=1}^4\sum_{j=1}^4\sum_{i=1}^4 a_{ijk} \\&= \sum_{k=1}^4\sum_{j=1}^4 (a_{1,jk} + a_{2,jk} + a_{3,jk} + a_{4,jk}) \\
        %     &= \sum_{k=1}^4 [(a_{1,1,k} + a_{2,1,k} + a_{3,1,k} + a_{4,1,k}) + (a_{1,2,k} + a_{2,2,k} + a_{3,2,k} + a_{4,2,k}) \\
        %     &+ (a_{1,3,k} + a_{2,3,k} + a_{3,3,k} + a_{4,3,k}) + (a_{1,4,k} + a_{2,4,k} + a_{3,4,k} + a_{4,4,k})]\\
        %     &= [(a_{1,1,1} + a_{2,1,1} + a_{3,1,1} + a_{4,1,1}) + (a_{1,2,1} + a_{2,2,1} + a_{3,2,1} + a_{4,2,1}) \\
        %     &+ (a_{1,3,1} + a_{2,3,1} + a_{3,3,1} + a_{4,3,1}) + (a_{1,4,1} + a_{2,4,1} + a_{3,4,1} + a_{4,4,1})] \\
        %     &+[(a_{1,1,2} + a_{2,1,2} + a_{3,1,2} + a_{4,1,2}) + (a_{1,2,2} + a_{2,2,2} + a_{3,2,2} + a_{4,2,2}) \\
        %     &+ (a_{1,3,2} + a_{2,3,2} + a_{3,3,2} + a_{4,3,2}) + (a_{1,4,2} + a_{2,4,2} + a_{3,4,2} + a_{4,4,2})]\\
        %     &+[(a_{1,1,3} + a_{2,1,3} + a_{3,1,3} + a_{4,1,3}) + (a_{1,2,3} + a_{2,2,3} + a_{3,2,3} + a_{4,2,3}) \\
        %     &+ (a_{1,3,3} + a_{2,3,3} + a_{3,3,3} + a_{4,3,3}) + (a_{1,4,3} + a_{2,4,3} + a_{3,4,3} + a_{4,4,3})]\\
        %     &+ [(a_{1,1,4} + a_{2,1,4} + a_{3,1,4} + a_{4,1,4}) + (a_{1,2,4} + a_{2,2,4} + a_{3,2,4} + a_{4,2,4}) \\
        %     &+ (a_{1,3,4} + a_{2,3,4} + a_{3,3,4} + a_{4,3,4}) + (a_{1,4,4} + a_{2,4,4} + a_{3,4,4} + a_{4,4,4})]
        % \end{align*}
    \end{solution}
    \item[5] What's wrong with the following derivation?
    \begin{equation*}
        \left(\sum_{j=1}^n a_j\right)\left(\sum_{k=1}^n \frac{1}{a_k}\right) = \sum_{j=1}^n\sum_{k=1}^n\frac{a_j}{a_k} = \sum_{k=1}^n\sum_{k=1}^n \frac{a_k}{a_k} = \sum_{k=1}^n n = n^2
    \end{equation*}
    \begin{solution}
        \begin{equation*}
            \sum_{j=1}^n\sum_{k=1}^n\frac{a_j}{a_k} \neq \sum_{k=1}^n\sum_{k=1}^n \frac{a_k}{a_k}
        \end{equation*}
        Because
        \begin{align*}
            \sum_{j=1}^n\sum_{k=1}^n\frac{a_j}{a_k} &= \sum_{j=1}^n\left(\frac{a_j}{a_1} + \cdots + \frac{a_j}{a_n}\right)\\
            &= \left(\frac{a_1}{a_1} + \cdots + \frac{a_1}{a_n}\right) + \cdots + \left(\frac{a_n}{a_1} + \cdots + \frac{a_n}{a_n}\right) 
        \end{align*}
        is not equal to
        \begin{equation*}
            \sum_{k=1}^n\sum_{k=1}^n \frac{a_k}{a_k} = \sum_{k=1}^n\sum_{k=1}^n \frac{a_k}{a_k} = \sum_{k=1}^n n = n^2
        \end{equation*}

    \end{solution}
\end{problems}

\end{document}
