\documentclass[a4paper,12pt]{article}
\usepackage{CJKutf8}
\usepackage{amsthm}
\usepackage{amsmath}
\usepackage{amssymb}
\usepackage{geometry}
\usepackage{tikz}
\usetikzlibrary{chains}
\usepackage{subfigure}
\usepackage{enumerate}
\usepackage{qrcode}

% 边距
\geometry{left=2.0cm,right=2.0cm,top=2.0cm,bottom=3.0cm}

\newtheorem{theorem}{Theorem}
\newtheorem{lemma}[theorem]{Lemma}
\newtheorem{proposition}[theorem]{Proposition}
\newtheorem{corollary}[theorem]{Corollary}
\newtheorem{exercise}{Exercise}
\newtheorem*{solution}{Solution}
\newtheorem{definition}{Definition}
\theoremstyle{definition}

\makeatletter \renewenvironment{proof}[1][Proof] {\par\pushQED{\qed}\normalfont\topsep6\p@\@plus6\p@\relax\trivlist\item[\hskip\labelsep\bfseries#1\@addpunct{.}]\ignorespaces}{\popQED\endtrivlist\@endpefalse} \makeatother
\makeatletter
\renewenvironment{solution}[1][Solution] {\par\pushQED{\qed}\normalfont\topsep6\p@\@plus6\p@\relax\trivlist\item[\hskip\labelsep\bfseries#1\@addpunct{.}]\ignorespaces}{\popQED\endtrivlist\@endpefalse} \makeatother

% 大题
\newenvironment{problems}{\begin{list}{}{\renewcommand{\makelabel}[1]{\textbf{##1}\hfil}}}{\end{list}}

% 小题
\newenvironment{steps}{\begin{list}{}{\renewcommand{\makelabel}[1]{\textbf{##1}\hfil}}}{\end{list}}

% 标题
\title{\small \underline{Mathematical Foundations of Computer Science}\\\Large Project 13}
\author{Zilong Li\\\small Student ID: 518070910095}
\date{\today}

\begin{document}
\maketitle

\noindent\textbf{Warmups}

\begin{problems}
   \item[7]Is (5.34) true also when $k < 0$?
   \begin{solution}
        It is also true that 
        \begin{equation*}
            r^{\underline{k}}\left(r-\frac{1}{2}\right)^{\underline{k}}= \frac{(2r)^{\underline{2k}}}{2^{2k}}, \text{when } k<0
        \end{equation*}
        The proof is as follows.
        \begin{align*}
            r^{\underline{k}}\left(r-\frac{1}{2}\right)^{\underline{k}} &= \frac{r}{r^{\overline{-k+1}}}\frac{r-\frac{1}{2}}{\left(r-\frac{1}{2}\right)^{\overline{-k+1}}} \\
            &= \frac{1}{(r+\frac{1}{2})(r+1)(r+\frac{3}{2})(r+2)\cdots(r-\frac{1}{2}-k)(r-k)} \\
            &= \frac{2^{-2k}}{(2r+1)(2r+2)\cdots(2r-1-2k)(2r-2k)}\\
            \frac{(2r)^{\underline{2k}}}{2^{2k}} &= \frac{2^{-2k}2r}{(2r)^{\overline{-2k+1}}}\\ 
            &= \frac{2^{-2k}}{(2r+1)(2r+2)\cdots(2r-2k)}
        \end{align*}
        And they are the same.
   \end{solution}
   \item[8]Evaluate 
   \begin{equation*}
       \sum_k\binom{n}{k}(-1)^k\left(1-\frac{k}{n}\right)^n
   \end{equation*}
   What is the approximate value of this sum, when $n$ is very large? Hint: The sum is $\Delta^n f(0)$ for some function $f$.
   \begin{solution}
       According to (5.40),
       \begin{align*}
           \Delta^n f(x) &= \sum_k\binom{n}{k}(-1)^{n-k}f(x+k) \\&=\sum_k\binom{n}{n-k}(-1)^{n-k}f(x+k)\\ &= \sum_k\binom{n}{k}(-1)^{k}f(x+n-k)
       \end{align*}
       In other word,
       \begin{equation*}
        \Delta^n f(0)  = \sum_k\binom{n}{k}(-1)^{k}f(n-k)
       \end{equation*}
       Compare to the formula to solve, the function $f$ is
       \begin{align*}
           f(n-k) &= \left(1-\frac{k}{n}\right)^n = \left(\frac{n-k}{n}\right)^n \\
           f(x) &= \left(\frac{x}{n}\right)^n
       \end{align*}
       As a result,
       \begin{align*}
        \sum_k\binom{n}{k}(-1)^k\left(1-\frac{k}{n}\right)^n &= \Delta^n f(0) \\&= \left(\Delta^n \left(\frac{x}{n}\right)^n\right)(0) \\&= \frac{(n-1)!}{n^n} \rightarrow 0, n\rightarrow +\infty
       \end{align*}
   \end{solution}
   \item[9] Show that the generalized exponentials of (5.58) obey the law
   \begin{equation*}
        \mathcal{E}_t(z) =\mathcal{E}(tz)^{1/t},\text{if }t\neq 0,
   \end{equation*}
    where $\mathcal{E}(z)$ is an abbreviation for $\mathcal{E}_1(z)$.
    \begin{proof}
        By (5.60),
        \begin{equation*}
            \mathcal{E}_t(z)^r = \sum_{k\geq 0} r \frac{(tk+r)^{k-1}}{k!} z^k
        \end{equation*}
        By assigning $r=t$, and notice that $t\neq 0$,
        \begin{align*}
            \mathcal{E}_t(z)^t &= \sum_{k\geq 0} t \frac{(tk+t)^{k-1}}{k!} z^k \\
                &= \sum_{k\geq 0} (k+1)^{k-1}\frac{(tz)^k}{k!} \\
                &= \mathcal{E}(tz)
        \end{align*}
        which is the same as the original formula.
        % From (5.59),
        % \begin{align*}
        %     \mathcal{E}(z)^{-t}\ln\mathcal{E}_t(z) &= z\\
        %     \mathcal{E}_t(z) &= \exp{z\mathcal{E}(z)^{t}}
        % \end{align*}
        % On the other hand, by (5.67)
        % \begin{align*}
        %     \mathcal{E}(tz)^{1/t} = (\exp{tz\mathcal{E}(tz)})^{1/t} = \exp{z\mathcal{E}(tz)}
        % \end{align*}
        % And it suffies to show that
        % \begin{equation*}
        %     \mathcal{E}(tz) = \mathcal{E}(z)^{t}
        % \end{equation*}
        % Again by (5.67),
        % \begin{equation*}
        %     \mathcal{E}(z)^t = \exp{tz\mathcal{E}(z)}
        % \end{equation*}
    \end{proof}
\end{problems}

\noindent\textbf{Basics}

\begin{problems}
   \item[14]Prove identity (5.25) by negating the upper index in Vandermonde's convolution (5.22).  Then show that another negation yields (5.26).
   \begin{proof}
    %    Vandermonde's convolution (5.22) gives
    %    \begin{equation*}
    %        \sum_k \binom{r}{m+k}\binom{s}{n-k} = \binom{r+s}{m+n}
    %    \end{equation*}
    %    negate the upper index of the first term,
    % %    and apply symmetry on the second term,
    %    \begin{equation*}
    %        \sum_k \binom{m+k-r-1}{m+k}\binom{s}{n-k} (-1)^k = \binom{r+s}{m+n}
    %    \end{equation*}
    %    assign $r=-m-1$,
    %    \begin{equation*}
    %     \sum_k \binom{2m+k}{m+k}\binom{s}{n-k} (-1)^k = \binom{s-m-1}{m+n}
    %     \end{equation*}
    %     \begin{align*}
    %         (-1)^{l+m}\binom{s-m-1}{l-m-n} &= (-1)^{l+m}\binom{s-m-1}{s-l+n-1} \\
    %         &= (-1)^{l+m}\frac{s-l+n}{s-m}\binom{s-m}{s-l+n} 
    %     \end{align*}
        For (5.25),
        \begin{align*}
            \sum_{k\leq l} \binom{l-k}{m}\binom{s}{k-n}(-1)^k  &= \sum_{k\leq l} \binom{l-k}{l-k-m}\binom{s}{k-n}(-1)^k &&\text{(symmetry)}\\
            &= \sum_{k\leq l} (-1)^{l-k-m}\binom{-m-1}{l-k-m}\binom{s}{k-n}(-1)^k && \text{(negation)} \\
            &= (-1)^{l-m}\sum_{k} \binom{s}{-n+k}\binom{-m-1}{l-m-k} && (m\geq 0) \\
            &= (-1)^{l-m} \binom{s-m-1}{l-m-n} && \text{(by 5.22)} \\
            &= (-1)^{l+m} \binom{s-m-1}{l-m-n} && ((-1)^{2m} = 1) 
        \end{align*}
        The step on $(m\geq 0)$ holds for the reason when $k>l$:
        \begin{equation}\label{eq:m}
            l-k-m \leq -1-m<0 \Rightarrow \binom{-m-1}{l-k-m} = 0,\quad (m\geq 0)
        \end{equation}
        For (5.26),
        \begin{align*}
            \sum_{0\leq k\leq l}\binom{l-k}{m}\binom{q+k}{n} 
            &= \sum_{0\leq k\leq l}\binom{l-k}{l-k-m}\binom{q+k}{q+k-n} && \text{(symmetry)} \\
            &= \sum_{0\leq k\leq l}\binom{-m-1}{l-k-m}\binom{-n-1}{q+k-n}(-1)^{l-k-m+q+k-n} &&\text{(nagation)}\\
            &= (-1)^{l-m+q-n} \sum_{k}\binom{-m-1}{l-k-m}\binom{-n-1}{q+k-n} && (m\geq 0,n\geq q) \\
            &= (-1)^{l-m+q-n} \binom{-m-n-2}{l-m+q-n} && (\text{by }5.22) \\
            &= \binom{l+q+1}{l-m+q-n} && \text{(negation)} \\
            &= \binom{l+q+1}{m+n-1} && \text{(symmetry)}
        \end{align*}
        and the step on $(m\geq 0,n\geq q)$ holds for
        \begin{equation*}
            k<0 \Rightarrow q+k-n\leq n+k-n = k<0 \Rightarrow\binom{-n-1}{q+k-n} = 0
        \end{equation*}
        the first term holds for the same reason as Eq. \eqref{eq:m}.
   \end{proof} 
   \item[15]What is $\sum_k\binom{n}{k}^3(-1)^k$? Hint: See (5.29).
   \begin{solution}
       By (5.29),
       \begin{equation*}
           \sum_{k}\binom{a+b}{a+k}\binom{b+c}{b+k}\binom{c+a}{c+k}(-1)^k = \frac{(a+b+c)!}{a!b!c!},\quad a,b,c\in\mathbb{N}.
       \end{equation*}
       when $n=2m(m\in\mathbb{N})$, the original formula could be deduced to
       \begin{align*}
        \sum_k\binom{n}{k}^3(-1)^k 
        % &= \sum_{k\geq 0}\binom{n}{k}^3(-1)^k &&\left({\left.\binom{n}{k}\right|_{k<0}=0}\right)\\
        &= \sum_k\binom{n}{n-k}^3(-1)^k &&\text{(symmetry)} \\
        &= \sum_k\binom{n}{n+k}^3(-1)^{-k} &&(k\rightarrow-k)\\
        &= \sum_k \binom{n}{n+k}\binom{n}{n+k}\binom{n}{n+k}(-1)^k(-1)^{-2k} &&((-1)^{-2k}=1)\\
        &= \frac{(m+m+m)!}{m!m!m!} &&\text{(by 5.29)}\\
        &= \frac{(3m)!}{m!^3}
       \end{align*}
       when $n=2m+1(m\in\mathbb{N})$, 
       \begin{equation*}
        \sum_k\binom{n}{k}^3(-1)^k = \sum_k\binom{n}{n-k}^3(-1)^k
       \end{equation*}
       expand both parts,
       \begin{align*}
           \binom{n}{0}^3 &+ \binom{n}{1}^3\times (-1) + \cdots + \binom{n}{n-1}^3 + \binom{n}{n}^3\times(-1) \\
           &= \binom{n}{n}^3  + \binom{n}{n-1}^3\times (-1) + \cdots + \binom{n}{1}^3+ \binom{n}{0}^3\times(-1)\\
           &= -\left[\binom{n}{0}^3 + \binom{n}{1}^3\times (-1) + \cdots + \binom{n}{n-1}^3 + \binom{n}{n}^3\times(-1)\right]
       \end{align*}
       then,
       \begin{equation*}
       2 \sum_k\binom{n}{k}^3(-1)^k = 0 \Rightarrow \sum_k\binom{n}{k}^3(-1)^k = 0 
       \end{equation*}
       when $n<0$,
       \begin{equation*}
           \binom{n}{k} = (-1)^k \binom{-n+k-1}{k}
       \end{equation*}
       and the original formula comes to 
       \begin{equation*}
        \sum_k\binom{n}{k}^3(-1)^k = \sum_{k\geq 0} \binom{-n+k-1}{k}^3
       \end{equation*}
       % How to do...?
       and $(-n+k-1)-k=-n-1\geq 0$, the symmetry property dispears. Every term is close to 1 when $n$ grows, so this is $+\infty$ when $n\rightarrow +\infty$.
    %    If the upper bound of $k$ is set to $m$, the deduction is as follows:
    %    \begin{align*}
    %     \sum_{k\leq m}\binom{n}{k}^3(-1)^k &= \sum_{0\leq k\leq m} \binom{-n+k-1}{-n-1}^3 \\
    %     &= \sum_{-n-1\leq k \leq -n+m-1}\binom{k}{-n+1}^3 = 
    %    \end{align*}
        In conclusion,
        \begin{equation*}
            \sum_k\binom{n}{k}^3(-1)^k = \begin{cases}
                \frac{(3m)!}{m!^3},&\text{if }n=2m(m\in\mathbb{N}),\\
                0,&\text{if }n=2m+1(m\in\mathbb{N}),\\
                +\infty,&\text{if }n<0.
            \end{cases}
        \end{equation*}
    \end{solution}
   \item[16]Evaluate the sum
   \begin{equation*}
       \sum_k\binom{2a}{a+k}\binom{2b}{b+k}\binom{2c}{c+k}(-1)^k
   \end{equation*}
   when $a, b, c$ are nonnegative integers.
   \begin{solution}
       \begin{equation*}
        \sum_k\binom{2a}{a+k}\binom{2b}{b+k}\binom{2c}{c+k}(-1)^k 
        % = \sum_k\binom{2a}{a-k}\binom{2b}{b-k}\binom{2c}{c-k}(-1)^k
        % &= \sum_k\binom{2a}{a-k}\binom{2b}{b-k}\binom{2c}{c-k}(-1)^{-k} &&((-1)^{2k}=1)
        = \sum_k \frac{(2a)!}{(a+k)!(a-k)!}\frac{(2b)!}{(b+k)!(b-k)!}\frac{(2c)!}{(c+k)!(c-k)!}(-1)^k
       \end{equation*}
       compared with (5.29),
       \begin{equation*}
        \sum_k\binom{a+b}{a+k}\binom{b+c}{b+k}\binom{c+a}{c+k}(-1)^k
        = \sum_k \frac{(a+b)!}{(a+k)!(b-k)!}\frac{(b+c)!}{(b+k)!(c-k)!}\frac{(c+a)!}{(c+k)!(a-k)!}(-1)^k
       \end{equation*}
       then,
       \begin{align*}
        \sum_k\binom{2a}{a+k}\binom{2b}{b+k}\binom{2c}{c+k}(-1)^k &= \frac{(2a)!(2b)!(2c)!}{(a+b)!(b+c)!(c+a)!}\sum_k\binom{a+b}{a+k}\binom{b+c}{b+k}\binom{c+a}{c+k}(-1)^k\\
        &= \frac{(2a)!(2b)!(2c)!}{(a+b)!(b+c)!(c+a)!} \frac{(a+b+c)!}{a!b!c!}
       \end{align*}
    %    expand both sides, assuming $m=\min{(a,b,c)}$,
    %    \begin{equation*}
    %     \sum_{k=-m}^m\binom{2a}{a+k}\binom{2b}{b+k}\binom{2c}{c+k}(-1)^k 
    %     = \sum_{k=-m}^{m}\binom{2a}{a-k}\binom{2b}{b-k}\binom{2c}{c-k}(-1)^k
    %    \end{equation*}
   \end{solution}
   \item[17]Find a simple relation between $\binom{2n-1/2}{n}$ and $\binom{2n-1/2}{2n}$.
   \begin{solution}
       By (5.35),
       \begin{equation*}
           \binom{r}{k}\binom{r-\frac{1}{2}}{k} = \left.\binom{2r}{2k}\binom{2k}{k}\right\slash 2^{2k}
       \end{equation*}
       and assigning $r=2n,k=n$,
       \begin{equation*}
           \binom{2n-\frac{1}{2}}{n} = \frac{\binom{4n}{2n}\binom{2n}{n}}{2^{2n}\binom{2n}{n}} = \left.\binom{4n}{2n}\right\slash 2^{2n}
       \end{equation*}
       assigning $r=2n,k=2n$,
       \begin{equation*}
           \binom{2n-\frac{1}{2}}{2n} = \frac{\binom{4n}{4n}\binom{4n}{2n}}{2^{2n}\binom{2n}{2n}} = \left.\binom{4n}{2n}\right\slash 2^{4n}
       \end{equation*}
       yields the simple relation of
       \begin{equation*}
           \binom{2n-\frac{1}{2}}{n} = 2^{2n}\binom{2n-\frac{1}{2}}{2n}
       \end{equation*}
   \end{solution} 
   \item[18]Find an alternative form analogous to (5.35) for the product
   \begin{equation*}
       \binom{r}{k}\binom{r-1/3}{k}\binom{r-2/3}{k}
   \end{equation*}
   \begin{solution}
    \begin{align*}
        r^{\underline{k}}&\left(r-\frac{1}{3}\right)^{\underline{k}}\left(r-\frac{2}{3}\right)^{\underline{k}}\\ &= r\left(r-\frac{1}{3}\right)\left(r-\frac{2}{3}\right)(r-1)\left(r-\frac{4}{3}\right)\left(r-\frac{5}{3}\right)\cdots (r-k+1)\left(r-k+\frac{2}{3}\right)\left(r-k+\frac{1}{3}\right) \\
        &=\frac{(3r)(3r-1)(3r-2)(3r-3)(3r-4)(3r-5)\cdots(3r-3k+3)(3r-3k+2)(3r-3k+1)}{3^{3k}}
        % \binom{r}{k}\binom{r-\frac{1}{3}}{k}\binom{r-\frac{2}{3}}{k} &= 
    \end{align*}
    divide both sides by $k!^3$,
    \begin{align*}
        \frac{r^{\underline{k}}}{k!}\frac{\left(r-\frac{1}{3}\right)^{\underline{k}}}{k!}\frac{\left(r-\frac{2}{3}\right)^{\underline{k}}}{k!} &= \frac{1}{3^{3k}}\frac{(3r)^{\underline{3k}}}{k!^3}\\
        \binom{r}{k}\binom{r-\frac{1}{3}}{k}\binom{r-\frac{2}{3}}{k} &= \frac{1}{3^{3k}}\frac{(3r)^{\underline{3k}}}{(3k)!}\frac{(3k)!}{k!^3}\\
        % = \frac{1}{3^{3k}}\frac{(3r)^{\underline{3k}}}{(3k)!}\frac{(3k)^{\underline{3k}}}{(3k)!}\frac{(3k)!}{k!^3}
        \binom{r}{k}\binom{r-\frac{1}{3}}{k}\binom{r-\frac{2}{3}}{k} &= \frac{1}{3^{3k}}\binom{3r}{3k}\frac{(3k)!}{2k!k!}\frac{(2k)^{\underline{k}}}{k!}\\
        \binom{r}{k}\binom{r-\frac{1}{3}}{k}\binom{r-\frac{2}{3}}{k} &= \frac{1}{3^{3k}}\binom{3r}{3k}\binom{3k}{2k}\binom{2k}{k}
    \end{align*}

   \end{solution}
\end{problems}

\end{document}