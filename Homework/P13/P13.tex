\documentclass[a4paper,12pt]{article}
\usepackage{CJKutf8}
\usepackage{amsthm}
\usepackage{amsmath}
\usepackage{amssymb}
\usepackage{geometry}
\usepackage{tikz}
\usetikzlibrary{chains}
\usepackage{subfigure}
\usepackage{enumerate}
\usepackage{qrcode}

% 边距
\geometry{left=2.0cm,right=2.0cm,top=2.0cm,bottom=3.0cm}

\newtheorem{theorem}{Theorem}
\newtheorem{lemma}[theorem]{Lemma}
\newtheorem{proposition}[theorem]{Proposition}
\newtheorem{corollary}[theorem]{Corollary}
\newtheorem{exercise}{Exercise}
\newtheorem*{solution}{Solution}
\newtheorem{definition}{Definition}
\theoremstyle{definition}

\makeatletter \renewenvironment{proof}[1][Proof] {\par\pushQED{\qed}\normalfont\topsep6\p@\@plus6\p@\relax\trivlist\item[\hskip\labelsep\bfseries#1\@addpunct{.}]\ignorespaces}{\popQED\endtrivlist\@endpefalse} \makeatother
\makeatletter
\renewenvironment{solution}[1][Solution] {\par\pushQED{\qed}\normalfont\topsep6\p@\@plus6\p@\relax\trivlist\item[\hskip\labelsep\bfseries#1\@addpunct{.}]\ignorespaces}{\popQED\endtrivlist\@endpefalse} \makeatother

% 大题
\newenvironment{problems}{\begin{list}{}{\renewcommand{\makelabel}[1]{\textbf{##1}\hfil}}}{\end{list}}

% 小题
\newenvironment{steps}{\begin{list}{}{\renewcommand{\makelabel}[1]{\textbf{##1}\hfil}}}{\end{list}}

% 标题
\title{\small \underline{Mathematical Foundations of Computer Science}\\\Large Project 13}
\author{Zilong Li\\\small Student ID: 518070910095}
\date{\today}

\begin{document}
\maketitle

\noindent\textbf{Warmups}

\begin{problems}
   \item[7]Is (5.34) true also when $k < 0$?
   \item[8]Evaluate 
   \begin{equation*}
       \sum_k\binom{n}{k}(-1)^k(1-\frac{k}{n})^n
   \end{equation*}
   What is the approximate value of this sum, when $n$ is very large? Hint: The sum is $\Delta^n f(0)$ for some function $f$.
   \item[9] Show that the generalized exponentials of (5.58) obey the law
   \begin{equation*}
        \mathcal{E}_t(z) =\mathcal{E}(tz)^{1/t},\text{if }t\neq 0,
   \end{equation*}
    where $\mathcal{E}(z)$ is an abbreviation for $\mathcal{E}_1(z)$.
\end{problems}

\noindent\textbf{Basics}

\begin{problems}
   \item[14]Prove identity (5.25) by negating the upper index in Vandermonde's con-volution (5.22).  Then show that another negation yields (5.26).
   \item[15]What is $\sum_k\binom{n}{k}^3(-1)^k$?Hint:See (5.29).
   \item[16]Evaluate the sum
   \begin{equation*}
       \sum_k\binom{2a}{a+k}\binom{2b}{b+k}\binom{2c}{c+k}(-1)^k
   \end{equation*}
   when $a, b, c$ are nonnegative integers.
   \item[17]Find a simple relation between $\binom{2n-1/2}{n}$ and $\binom{2n-1/2}{2n}$.
   \item[18]Find an alternative form analogous to (5.35) for the product
   \begin{equation*}
       \binom{r}{k}\binom{r-1/3}{k}\binom{r-2/3}{k}
   \end{equation*}.
\end{problems}

\end{document}